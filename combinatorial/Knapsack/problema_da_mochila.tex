\documentclass{article}
\usepackage[portuguese]{babel}
\usepackage[a4paper, top=3cm, bottom=2cm, left=2.5cm, right=2.5cm]{geometry}
\usepackage{blindtext}
\usepackage{datetime}
\usepackage{csvsimple}
\usepackage[T1]{fontenc}
\usepackage{graphicx} 
\usepackage{minted}
\usepackage{xcolor}
\usepackage{tcolorbox}
\usepackage{indentfirst}
\usepackage{fancyhdr}  
\usepackage{caption}
\usepackage{enumitem}
\usepackage{titlesec}
\usepackage{titletoc}
\usepackage{hyperref}

\newcommand{\annotationbox}[3]{
    \definecolor{currentcol}{named}{#3} 
    \begin{tcolorbox}[colback=currentcol!5!white, colframe=currentcol!75!black, title={#1}]
        {#2}
    \end{tcolorbox}
}

\newcommand{\coded}[5]{
    \definecolor{Gray}{gray}{0.9}
    \begin{minipage}{\textwidth}
        \captionof{listing}{{#5}}
        \inputminted[
            frame=lines,
            framesep=2mm,
            baselinestretch=1.2,
            firstline={#3},
            lastline={#4},
            bgcolor=Gray,
            linenos]{#1}{#2}
    \end{minipage}
}
\renewcommand\listingscaption{Código}
\renewcommand\listoflistingscaption{Lista de Códigos}

\newcommand{\csvtable}[3][]{%
  \begin{table}[H]
    \centering
    \csvautotabular{#2}
    \caption{#3}
    \IfNoValueF{#1}{\label{tab:#1}}
  \end{table}%
}
%uniform ref
\newcommand{\figref}[1]{figura \ref{#1}}
\newcommand{\tabref}[1]{tabela \ref{#1}}
\newcommand{\eqnref}[1]{equação (\ref{#1})}

\newcommand{\theauthor}{Vinícius Sousa Dutra (13686257)}
\newcommand{\thetitle}{Problema da Mochila}

\newcommand{\addimage}[3][100mm]{
    \begin{figure}[H]
        \centering
        \includegraphics[width={#1}]{{#2}}
        \caption{{#3}}
    \end{figure}    
}   

\pagestyle{fancy}  
\fancyhf{}  
\rhead{\theauthor} 
\lhead{Problema da Mochila}  
\cfoot{\thepage} 


\begin{document}



\begin{titlepage}
    \begin{center}
        \begin{figure}[htb!]
            \centering
            \includegraphics[width=150mm]{images/ifsc_logo.jpg}
        \end{figure}
        \vspace{20pt}
        \LARGE{\textbf{Universidade de São Paulo}}\\
        \LARGE{Instituto de Física de São Carlos}\\

        \vspace{150pt}

        
        \LARGE{\textbf{{\thetitle}}} 
        \\ 
        \textsc{\LARGE Otimização combinatória}
        \\
        
        
        \vspace{125pt}
        \begin{minipage}{\textwidth}
            \begin{flushleft} \large
                \textbf{Professor:}\\
                Luciano da Fontoura Costa\\[0.8cm]
                \textbf{Aluno:}\\
                \theauthor
                
                \end{flushleft}
                \end{minipage}\\[1 cm]
        \vspace{30pt}
        \vspace{\fill}  
        \Large {\today}

    \end{center}
\end{titlepage}

\newpage
\tableofcontents
\listoffigures
\listoftables
\listoflistings
\newpage
\section{Definição}

Formalmente podemos considerar o problema 
como sendo
\begin{equation}
    maximize  \, \, \sum v_ix_i
\end{equation}
\begin{equation}
    subject \, to \, \, \sum w_ix_i \le W
\end{equation}

Considerando que $v_i$ seja um conjunto de valores,
$w_i$ seja um conjunto de pesos, $W$ seja o
peso total e $x_i$ seja o número de itens escolhidos,
temos um valor máximo $c_i$ associado a cada item.


Dado que todas as demais variáveis são fixas, 
a variável real no contexto do problema é $x_i$.
O conjunto de soluções cresce de maneira exponencial, 
quando consideramos o caso mais simples 
no qual apenas um exemplar de cada item é permitido,
o número de soluções viáveis é dado por 
$2^{n-1}$, onde $n$ é o numero total de itens. 


Isso ocorre devido ao fato de que o número total de 
soluções possíveis é $2^n$, e para cada solução 
não viável $X$, a sua solução complementar 
$\overline{X}$ é uma solução viável.

No caso que iremos estudar com $N=1000$ e com o 
número de exemplares variados, o limite inferior 
de soluções é de $10^{301}$. Um limite superior para 
o conjunto específico de dados usados foi calculado 
realizando a seguinte operação

\begin{equation}
    \log(L_{superior})=\sum   \log(c_i+1)
\end{equation}
\begin{equation}
    L_{superior} \approx 10^{734}
\end{equation}

\section{Geração de Dados}
Os dados são gerados de maneira aleatória
usando esse pequeno script Python apresentado abaixo

\coded{python3}{generate_csv.py}{1}{9}{Geração do csv}
\section{Busca aleatória}
A busca aleatória consiste em criar uma sequência 
de $N$ zeros. Em seguida, são escolhidos 
índices de forma aleatória, e para cada índice, 
o espaço vazio é preenchido com um número
aleatório de itens selecionados. 
O processo é pausado até que o peso 
máximo seja atingido. Em cada iteração, 
a melhor solução encontrada é guardada.



\coded{python3}{random_search.py}{26}{46}{Busca aleatória}

\section{Simulated Annealing}

O Simulated Annealing utilizado é semelhante ao 
utilizado no Problema do Caixeiro Viajante 
(TSP - Traveling Salesman Problem). 
As soluções vizinhas são encontradas realizando 
três operações básicas.

\begin{itemize}
    \item Adição aleatória de item
    \item Remoção aleatória de item
    \item Swap de índices
\end{itemize}

Para preservar a viabilidade da solução, 
é sempre verificado se a operação é permitida.
No final de ambos os códigos, é verificado se 
a melhor solução é viável.

\coded{python3}{simulated_annealing.py}{46}{78}{Simulated Annealing}

\section{Conclusão}
Ambos os algoritmos foram testados no mesmo conjunto 
de dados com o mesmo número de iterações. 
Observou-se que, para poucas iterações, os algoritmos 
são equivalentes. Nas figuras abaixo, observou-se 
que o método aleatório teve um desempenho um pouco
melhor do que o Simulated Annealing (SA).


\addimage{images/plot_SA_mil.jpg}{Simulated Annealing, $10^3$ iterações}
\addimage{images/plot_random_search_mil.jpg}{Busca aleatória, $10^3$ iterações}

No entanto, para $10^5$ iterações, é visível uma 
estagnação da busca aleatória. Isso ocorre porque, 
como discutido, o conjunto de soluções é muito 
grande para encontrar boas soluções aleatoriamente. 
Nesse caso, o Simulated Annealing (SA) encontrou 
soluções cerca de 3 vezes melhores.

Também foi observada uma diferença significativa 
no tempo de execução, em que o SA teve um tempo
de execução de $8s$ e a busca aleatória de $120s$. 
Como o SA altera soluções já existentes em vez de 
criar novas em cada iteração, ele acaba executando 
mais rapidamente.

\addimage{images/plot_SA.jpg}{Simulated Annealing, $10^5$ iterações}
\addimage{images/plot_random_search.jpg}{Busca aleatória, $10^5$ iterações}
\end{document}
